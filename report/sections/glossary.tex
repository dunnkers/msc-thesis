
%%%%%%%%%%%%%%%%%%%%%%%%%%%%%%%%%%%%%%%%%%%%%%%%%%%%%%%%%%%%%%%%%%%%
%%--------------------------GLOSSARY------------------------------%%
%%%%%%%%%%%%%%%%%%%%%%%%%%%%%%%%%%%%%%%%%%%%%%%%%%%%%%%%%%%%%%%%%%%%

\makenoidxglossaries


% ACRONYMS
\newacronym{rfe}{RFE}{Recursive Feature Elimination}
\newacronym{cv}{CV}{Cross-Validation}
\newacronym{dt}{DT}{Decision Tree}
\newacronym{lasso}{LASSO}{Least Absolute Shrinkage and Selection Operator}
\newacronym{pca}{PCA}{Principle Component Analysis}
\newacronym{lda}{LDA}{Linear Discriminant Analysis}
\newacronym{knn}{k-NN}{k- Nearest Neighbors}
\newacronym{nb}{NB}{Naïve Bayes}
\newacronym{svd}{SVD}{Singular Value Decomposition}
\newacronym{ml}{ML}{Machine Learning}
\newacronym{ai}{AI}{Artificial Intelligence}
\newacronym{pcc}{PCC}{Pearson product-moment Correlation Coefficient}
\newacronym{hpc}{HPC}{High-Performance Computing}



% DESCRIPTIONS
\newglossaryentry{rq}
{
    name={RQ},
    description={``RQ (Redis Queue) is a simple Python library for queueing jobs and processing them in the background with workers. It is backed by Redis and it is designed to have a low barrier to entry.'' - \href{source}{https://python-rq.org/}}
}

\newglossaryentry{apriori}
{
    name={apriori},
    description={\textit{Apriori} information refers to any possible information about the dataset feature importances one has before running any learning algorithms. Common ways to obtain such information are human domain experts or the usage of synthetic datasets}
}

\newglossaryentry{ib1}
{
  name={IB1},
  description={Instance-based classification algorithm originally proposed by Aha et al. (1991) \citep{aha_instance-based_1991}},
  first={\glsentrytext{ib1}},
  plural={IB1},
  firstplural={\glsentrytext{ib1}}
}

\newglossaryentry{part}
{
  name={PART},
  description={Partial Decision Tree algorithm},
  first={\glsentrytext{part}},
  plural={PART},
  firstplural={\glsentrytext{part}}
}

\newglossaryentry{rba}
{
  name={RBA},
  description={Relief-Based Feature Selection Algorithms},
  first={\glsentrydesc{rba} (\glsentrytext{rba})},
  plural={RBAs},
  firstplural={\glsentrydesc{rba}s (\glsentryplural{rba})}
}

\newglossaryentry{svm}
{
  name={SVM},
  description={Support Vector Machine},
  first={\glsentrydesc{svm} (\glsentrytext{svm})},
  plural={SVMs},
  firstplural={\glsentrydesc{svm}s (\glsentryplural{svm})}
}